\documentclass{article}
\usepackage{hyperref}

\title{AUNSW IT Subcommittee Handbook}
\date{2025}
\author{Rin}

\hypersetup{
    colorlinks=true,
    linkcolor=magenta,
}

\begin{document}
\renewcommand{\labelenumi}{(\roman{enumi})}
\renewcommand{\labelenumii}{(\alph{enumii})}
\maketitle
Congratulations on making it into the 2025 AUNSW subcom team! You are now my slave.

\textbf{AUNSW} is a society which belongs to the \textbf{University of New South Wales} and operates under the club structure of \textbf{ARC @ UNSW}. Our society aims to create a fun and supportive community for people that are interested in all things anime!

I stole the majority of this guide so thanks to whoever actually made it!

\section{Chain of Command}
It is important to have a chain of command in any organization. In this section, you will find out a bit more about the structure of our society. It is important that you familiarize yourself with the people and roles.
\subsection{Top 5}
The ‘Top 5’ are the executives in the highest positions in the society. This may sound intimidating, but they are all really chill. The positions are as follows:
\begin{itemize}
\item President (Ethan)
\item Vice President (Fidelina (Koi))
\item External Relations (Felix)
\item Secretary/Grievance Officer (Nyil (Raoli))
\item Treasurer (Nolan)
\end{itemize}
\subsection{Portfolios}
There are four main portfolios in the society - Events, Arts, Marketing/Media, and IT.
Events organises and runs society events, and is managed by one Lead Events Director and three Co-Events Directors.
\begin{itemize}
\item Events Director (Yiming)
\item Co-Events Director (Vincent)
\item Co-Events Director (Lia)
\item Co-Events Director (Sam)
\end{itemize}
Arts draws assets for the society’s event banners, as well as design merchandise for sale at SMASH or other events. Like Events, it is managed by one Lead Arts Director and three Co-Arts Directors. They look over a team of 6 subcommittee members.
\begin{itemize}
\item Arts Director (Daniel)
\item Co-Arts Director/Admin (Lachlan)
\item Co-Arts Director (Ange)
\item Co-Arts Director (Katherine)
\end{itemize}
Marketing/Media does marketing for events and maintains media such as photography and videography.
\begin{itemize}
\item Marketing Director (Ethan)
\item Media Director (Raine)
\end{itemize}
IT manages the society's website, discord server and discord bots. We also aim to improve our internal processes.
\begin{itemize}
\item IT Director (Rin (Me))
\end{itemize}
\subsection{Arc}
AnimeUNSW is an Arc Affiliated Club, which means we need to obey their laws and regulations to keep getting society funding. I would recommend reading the \href{https://www.arc.unsw.edu.au/clubs/clubshandbook}{Arc Clubs Handbook} to familiarize yourself with how it works.
\section{Time Commitment}
Depending on your uni/work schedules, a set time commitment will not be imposed. However, it is an expectation that you will spend an average of 2-5 hours weekly (changes significantly with seasonal projects)
If you are busy or unavailable for a period of time (due to assignments, travel, etc.), that is acceptable - please let the team know in advance so we can cover you and not assign too many tasks.
If you are feeling burnt out, it is important to take a break. We’ll be on the lookout for signs, but please let one of your execs know if you need a breather so we can care for you most effectively.
\section{Communication}
Before even going into WHAT you do, it is important to cover HOW you do it. Remember, you are part of a team and good communication is crucial to running a smooth marketing operation. Specifically, this would include:
\subsubsection*{Answer when you are messaged}
Don’t just leave messages on read. Even a thumbs up to express acknowledgement can go a long way.
\subsubsection*{Ask questions when unsure}
It’s a lot better to avoid mistakes that arise from miscommunication or assumptions and asking questions can achieve this.
\subsubsection*{Apologize when you make a mistake}
Unfortunately, making mistakes is unavoidable. Directly acknowledge fault to anyone involved, even if the mistake is small.
\subsubsection*{Keep the others informed}
If any issues arise, personal or task related, let the other committee members know as soon as possible
\subsubsection*{Be nice}
Even if there is someone on the team who you don’t mix well with, treat them with respect so you work together and not against each other. Understandably, some of our members are not so sociable. Please treat them with respect regardless.
\subsubsection*{Response Times}
Whilst there is no expectation of instant responses, responses within 1-2 business days would be preferred for smooth operations. Of course during busier times like exam or assignment season, we can be more flexible to accommodate your heavier workloads.
\subsubsection*{Let others know if you need help}
If you are struggling with anything please let the rest of the team know! Don’t suffer in silence and remember that our responsibilities are shared.
\section{Weekly Duties}
\subsection{Meetings}
Meetings are the lifeblood of any club committee. Their purpose is to discuss, inform and create actionable tasks that eventually lead to successful events.
Meetings are usually “chaired” or led by a single person, while a scribe writes the minutes. For your first few meetings, you will not be expected to chair meetings or write minutes.
\subsubsection{Attending Meetings}
If you can attend a meeting, please do. A mobile connection to the meeting can work in a pinch. When you attend a meeting, if you have any input to the discussion, please voice yourself. Without input, meetings become a passive task that can quickly become as mundane as a lecture.
\subsubsection{Minutes}
Minutes are the written version of a meeting. Ideally, they are used as reference in case you forgot something or for researching past events. If you are unable to attend a meeting, please read the minutes as soon as possible.

The outline for minutes should be written before the meeting itself, which will fill in the gaps. Sometimes, not everything will be addressed within the meeting and should be noted for further discussion.

Try to use consistent formatting that improves readability. Different agendas will be separated into different rows in a table, whilst sub-agendas will be separated by dot point indentations. Actionables should be highlighted to easily distinguish them.

If the meetings are becoming boring, you can volunteer to write the minutes! Please don’t write the minutes if you are chairing the meeting though. Try to capture as much detail as possible when writing the minutes. It’s much easier to edit out redundant information than to remember details.
\subsubsection{Actionables}
Actionables refers to the tasks that committee members have to do outside of meetings. These are usually located at the end of the minutes. It is preferable to have another person dedicated to filling out the actionables alongside the person writing the minutes just so the person writing minutes does not have to stay behind after the meeting.

Actionables should also be posted on the team chat. Remember to let the others know when you complete your actionables to avoid angry execs trying to chase you up. A good practice is to log all your actionables in your calendar app or any productivity app. (Do this for your uni stuff too)

Sometimes tasks may arise that are not covered in the meetings. These could be a result of oversights, accidents or (an exec’s) assumption that regular tasks don’t need to be written in actionables. You may even be required to take over tasks originally assigned to others. It can be annoying, but please treat these as normal actionables - stay on top of them and keep the others informed of your progress.
\section{IT Portfolio}
The IT portfolio focuses mainly on our website and Discord bots. We also are the official Discord mods, so have fun with that...! I'll try to split tasks evenly between projects, so that you don't feel like you're doing the same thing forever; let me know if this isn't working out for whatever reason.
\subsection{The Website}
Our website is currently hosted at \url{https://animeunsw.net} and is in the middle of a complete overhaul. It uses Next.js, a React based framework for both frontend and backend development. In addition, we use PostgreSQL as a database. As such it is requested of you to become familiar with these technologies.
\subsection{The Discord Bot(s)}
On the official AnimeUNSW Discord server, we use a bot called Ibi Bot. At the writing of this guide she only has verification related commands, but we intend to expand this further in the future. She runs on discord.py, a Python library for interacting with the Discord API. Please ensure that you are familiar with the library and its documentation.

We also intend to create a Discord bot for our internal server, in order to provide a more seamless experience for our execs and subcommittee.
\subsection{GitHub}
I will be inviting you to the AnimeUNSW GitHub organization. This is where all of our code is hosted, and where you will be able to contribute to our projects. Please ensure that you are familiar with the organization's guidelines to contributing to our projects. In addition, our planned features and ideas will be hosted here, so that they are available for future years. Thus, our actionables and timeline will also be hosted here, so make sure you are familiar with Git and GitHub.
\subsection{Anything Else?}
Up to you! We're very open to any ideas, so feel free to suggest anything you believe would benefit the society, externally or internally. I'm terminally online, so feel free to message me anytime!
\section{Miscellaneous}
\subsection{Attending Events}
Even though you won’t usually have an active role when attending events, it is important to remember that you are still a representative of the society, and should behave accordingly.

It is always important to make sure members feel welcomed at our events. As such, it is up to you to help create an environment where members feel like they belong. Try not to just close yourself off in a small group of friends, and just chat to other members! To some of you, this may come easily, but to others it may be more difficult. It is recommended that you talk to other people in a group of two (i.e. you and another subcom member) so it’s less intimidating for you and the members. Just try your best!
\subsection{Bondage}
Now when you hear the term ‘bondage’ in the society, it usually does NOT refer to BDSM, but instead, social events to help us bond together as a team. This includes day outings and toad trips.

\end{document}
